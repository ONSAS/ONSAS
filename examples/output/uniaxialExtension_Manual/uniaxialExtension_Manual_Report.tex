\documentclass[a4paper,10pt]{article} 
\usepackage[a4paper,margin=20mm]{geometry} 
\usepackage{longtable} 
\usepackage{float} 
\usepackage{adjustbox} 
\usepackage{graphicx} 
\usepackage{color} 
\usepackage{booktabs} 
\aboverulesep=0ex 
\belowrulesep=0ex 
\usepackage{array} 
\newcolumntype{?}{!{\vrule width 1pt}}
\usepackage{fancyhdr} 
\pagestyle{fancy} 
\fancyhf{} 
\lhead{Report ONSAS version 0.1.10} 
\rhead{Problem: uniaxialExtensionManual } 
\lfoot{Date: \today} 
\rfoot{Page \thepage} 
\renewcommand{\footrulewidth}{1pt} 
\renewcommand{\headrulewidth}{1.5pt} 
\setlength{\parindent}{0pt} 
\usepackage[T1]{fontenc} 
\usepackage{libertine} 
\usepackage{arydshln} 
\definecolor{miblue}{rgb}{0,0.1,0.38} 
\usepackage{titlesec} 
\titleformat{\section}{\normalfont\Large\color{miblue}\bfseries}{\color{miblue}\sectionmark\thesection}{0.5em}{}[{\color{miblue}\titlerule[0.5pt]}] 

\begin{document} 
\begin{center} 
\textbf{ONSAS v.0.1.10 analysis report \\ Problem: uniaxialExtensionManual} 
\end{center} 

This is an ONSAS automatically-generated report with part of the results obtained after the analysis. The user can access other magnitudes and results through the GNU-Octave/MATLAB console. The code is provided AS IS \textbf{WITHOUT WARRANTY of any kind}, express or implied.

\section{Analysis results}

\begin{longtable}{cccccccc} 
$\#t$ & $ \lambda(t)$ & its & $\| RHS \|$ & $\| \Delta u \|$ & flagExit  & npos & nneg  \\ \hline 
 \endhead 
\hdashline
   1 &  0.00e+00 &    0 &           &           &  0 &   0 &   0 \\ 
     &           &    1 &  9.49e-02 &  6.52e-01 &    &     &     \\ 
     &           &    2 &  6.70e-03 &  1.48e-01 &    &     &     \\ 
     &           &    3 &  5.28e-05 &  1.44e-02 &    &     &     \\ 
     &           &    4 &  3.10e-09 &  1.07e-04 &    &     &     \\ 
     &           &    5 &  2.39e-17 &  6.51e-09 &    &     &     \\ 
\hdashline
   2 &  1.00e-01 &    5 &           &           &  1 &   0 &   0 \\ 
     &           &    1 &  3.69e-02 &  3.96e-01 &    &     &     \\ 
     &           &    2 &  7.87e-04 &  4.83e-02 &    &     &     \\ 
     &           &    3 &  4.82e-07 &  1.35e-03 &    &     &     \\ 
     &           &    4 &  1.65e-13 &  7.47e-07 &    &     &     \\ 
\hdashline
   3 &  2.00e-01 &    4 &           &           &  1 &   0 &   0 \\ 
     &           &    1 &  2.34e-02 &  3.13e-01 &    &     &     \\ 
     &           &    2 &  2.37e-04 &  2.51e-02 &    &     &     \\ 
     &           &    3 &  3.28e-08 &  3.48e-04 &    &     &     \\ 
     &           &    4 &  5.74e-16 &  4.21e-08 &    &     &     \\ 
\hdashline
   4 &  3.00e-01 &    4 &           &           &  1 &   0 &   0 \\ 
     &           &    1 &  1.74e-02 &  2.72e-01 &    &     &     \\ 
     &           &    2 &  1.03e-04 &  1.57e-02 &    &     &     \\ 
     &           &    3 &  4.85e-09 &  1.32e-04 &    &     &     \\ 
     &           &    4 &  8.78e-17 &  5.64e-09 &    &     &     \\ 
\hdashline
   5 &  4.00e-01 &    4 &           &           &  1 &   0 &   0 \\ 
     &           &    1 &  1.41e-02 &  2.48e-01 &    &     &     \\ 
     &           &    2 &  5.47e-05 &  1.10e-02 &    &     &     \\ 
     &           &    3 &  1.04e-09 &  5.59e-05 &    &     &     \\ 
     &           &    4 &  1.41e-17 &  1.32e-09 &    &     &     \\ 
\hdashline
   6 &  5.00e-01 &    4 &           &           &  1 &   0 &   0 \\ 
     &           &    1 &  1.21e-02 &  2.35e-01 &    &     &     \\ 
     &           &    2 &  3.40e-05 &  8.68e-03 &    &     &     \\ 
     &           &    3 &  2.73e-10 &  2.44e-05 &    &     &     \\ 
     &           &    4 &  1.76e-16 &  1.98e-10 &    &     &     \\ 
\hdashline
   7 &  6.00e-01 &    4 &           &           &  1 &   0 &   0 \\ 
     &           &    1 &  1.08e-02 &  2.29e-01 &    &     &     \\ 
     &           &    2 &  2.53e-05 &  8.14e-03 &    &     &     \\ 
     &           &    3 &  4.68e-10 &  4.92e-05 &    &     &     \\ 
     &           &    4 &  3.51e-16 &  3.06e-09 &    &     &     \\ 
\hdashline
   8 &  7.00e-01 &    4 &           &           &  1 &   0 &   0 \\ 
     &           &    1 &  1.01e-02 &  2.30e-01 &    &     &     \\ 
     &           &    2 &  2.52e-05 &  9.55e-03 &    &     &     \\ 
     &           &    3 &  2.54e-09 &  1.33e-04 &    &     &     \\ 
     &           &    4 &  6.02e-17 &  2.18e-08 &    &     &     \\ 
\hdashline
   9 &  8.00e-01 &    4 &           &           &  1 &   0 &   0 \\ 
     &           &    1 &  9.80e-03 &  2.40e-01 &    &     &     \\ 
     &           &    2 &  3.72e-05 &  1.38e-02 &    &     &     \\ 
     &           &    3 &  1.75e-08 &  3.75e-04 &    &     &     \\ 
     &           &    4 &  4.77e-15 &  1.95e-07 &    &     &     \\ 
\hdashline
  10 &  9.00e-01 &    4 &           &           &  1 &   0 &   0 \\ 
     &           &    1 &  1.02e-02 &  2.62e-01 &    &     &     \\ 
     &           &    2 &  8.24e-05 &  2.41e-02 &    &     &     \\ 
     &           &    3 &  2.25e-07 &  1.45e-03 &    &     &     \\ 
     &           &    4 &  1.45e-12 &  3.66e-06 &    &     &     \\ 
\hdashline
  11 &  1.00e+00 &    4 &           &           &  1 &   0 &   0 \\ 
     &           &    1 &  1.21e-02 &  3.12e-01 &    &     &     \\ 
     &           &    2 &  3.23e-04 &  5.57e-02 &    &     &     \\ 
     &           &    3 &  1.38e-05 &  1.23e-02 &    &     &     \\ 
     &           &    4 &  1.84e-08 &  4.51e-04 &    &     &     \\ 
     &           &    5 &  3.23e-14 &  6.01e-07 &    &     &     \\ 
\hdashline
  12 &  1.10e+00 &    5 &           &           &  1 &   0 &   0 \\ 
 
\caption{Output of incremental analysis.}
\end{longtable}

\newpage 

\end{document}